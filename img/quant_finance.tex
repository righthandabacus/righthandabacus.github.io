\documentclass[9pt,twocolumn]{extarticle}
%\documentclass{IEEEtran}
\usepackage{amsmath}
\usepackage{amssymb}
\usepackage[margin=15mm]{geometry}
\usepackage{unicode-math} % https://tex.stackexchange.com/questions/96024/how-to-use-system-font-for-equation-in-xelatex
%\setmainfont[AutoFakeSlant,AutoFakeBold]{TeX Gyre Pagella}
\defaultfontfeatures{Mapping=tex-text,Scale=MatchLowercase} %,Ligatures={Common,Rare}}
\setmainfont{TeX Gyre Pagella} % Adventor Bonum Cursor Heros Pagella Termes Schola Chorus
%\setmathfont{Asana Math}
\setmathfont{TeX Gyre Pagella Math} % unicode-math: Cambria Math, Latin Modern Math, TeX Gyre Pagella Math, Asana Math, Neo Euler, STIX, XITS Math

%\usepackage{parskip}
\setlength{\parindent}{0pt}
\setlength{\parskip}{0pt}
\setlength{\parsep}{0pt}
%\setlength{\headsep}{0pt}
%\setlength{\topskip}{0pt}
%\setlength{\topmargin}{0pt}
%\setlength{\topsep}{0pt}
\setlength{\partopsep}{0pt}
\setlength{\abovedisplayskip}{1pt}
\setlength{\belowdisplayskip}{1pt}
\setlength{\abovedisplayshortskip}{0pt}
\setlength{\belowdisplayshortskip}{0pt}
\usepackage[compact]{titlesec}
\titlespacing{\section}{0pt}{*1}{*0}
\titlespacing{\subsection}{0pt}{*1}{*0}
\titlespacing{\subsubsection}{0pt}{*1}{*0}

\usepackage{pgfplots}
\pgfplotsset{
    compat=1.14,
    axis lines=center,
    scale only axis=true,
    every axis y label/.style={
        at={(ticklabel* cs:1)}, anchor=south, font=\tiny
    },
    every axis x label/.style={
        at={(ticklabel* cs:1)}, anchor=west, font=\tiny
    },
    every axis title/.append style={
        at={(0,1)},
        anchor=north west,
        font=\footnotesize,
        xshift=3pt, yshift=0pt
    },
    every tick label/.append style={
        font=\tiny
    },
    every axis legend/.append style={
        font=\tiny,
        fill=none,
        draw=none
    },
    xtick distance=1,
    restrict y to domain=-100:100,
    unbounded coords=jump,
}
\tikzset{>=latex}
\DeclareMathOperator{\E}{E}
\DeclareMathOperator{\Var}{Var}
\DeclareMathOperator{\Cov}{Cov}

\usepackage{enumitem}
\setlist{nosep} % or \setlist{noitemsep} to leave space around whole list

\usepackage{fancyhdr}
\pagestyle{fancy}
\fancyhf{}
\renewcommand{\headrulewidth}{0pt}
\fancyfoot[R]{\thepage}

\usepackage{marginnote}

%%%%%%%%%%%%%%%%%%%%%%%%%%%%%%%%%%%%%%%%%%%%%%%%%

\begin{document}

\section{Bond \& Time Value of Money}

Compound interest, and continuous compounding:
\begin{align*}
FV &= PV\left(1+\frac{r}{n}\right)^{nt}  &  PV &= FV\left(1+\frac{r}{n}\right)^{-nt} \\
P_t &= P_0 e^{rt}                        &  P_o &= P_t e^{-rt}
\end{align*}

Cash flow, with interest rate $r_i$ at period $i$:
$$ PV = \sum_{i=1}^n \frac{CF_i}{(1+r_i)^i} $$

Excel calculates for constant $CF$ and $r$ with \texttt{=PV(r,n,CF)}.

\emph{Perpetual bond}: Fixed payment $CF$ each period forever, and assumed $r$ constant, $n\to\infty$:
$$ PV = \sum_{i=1}^{\infty} \frac{CF}{(1+r)^i} = \frac{CF/(1+r)}{1-\frac{1}{1+r}} = \frac{CF}{r} $$
Rate $r$ in perpetual bond is the hurdle rate of a corp such that it is the min
IRR required for any investment.

\emph{Gordon growth model}: payment grow at a constant rate $CF_i = CF_{i-1}(1+g)$. For finite term:
$$ PV = \sum_{i=1}^n \frac{CF_0(1+g)^i}{(1+r_i)^i} = CF_0\sum_{i=1}^n \frac{(1+g)^n}{(1+r)^n} $$
As $n\to\infty$, $ PV =  CF_0/(r-g) $

\subsection*{Bond pricing}
Bond with coupon $C$, constant discount rate $y$ and repaid \$1 after $n$ terms,
$$ PV = \sum_{i=1}^n \frac{C}{(1+y)^i} + \frac{1}{(1+y)^n}
      = \frac{C}{y} \left(1-\frac{1}{(1+y)^n}\right) + \frac{1}{(1+y)^n}.
$$
If PV is the price, $y$ here is the yield for the bond. Price sensitivity to the yield:
\begin{gather*}
\frac{\partial P}{\partial y}
= \frac{\partial}{\partial y} \left(\sum_{i=1}^n \frac{C}{(1+y)^i} + \frac{Par}{(1+y)^n}\right)
= \sum_{i=1}^n \frac{-iC}{(1+y)^{i+1}} + \frac{-nPar}{(1+y)^{n+1}} \\
\frac{1}{P}\frac{\partial P}{\partial y}
= -\frac{1}{1+y}\frac{1}{P}\left( \sum_{i=1}^n \frac{iC}{(1+y)^i} + \frac{nPar}{(1+y)^n} \right)
\end{gather*}
which we define the Macauley duration and modified duration as
\begin{align*}
D_{\textrm{Mac}} &= \frac{1}{P}\left( \sum_{i=1}^n \frac{iC}{(1+y)^i} + \frac{nPar}{(1+y)^n} \right) \\
D_{\textrm{mod}} &= \frac{1}{1+y}D_{\textrm{Mac}}
\end{align*}
respectively. Then px sensitivity $\frac{1}{P}\frac{\partial P}{\partial y} = -D_{\textrm{mod}}$.
Given $D_{\textrm{mod}}$, a bond at $P$ will see price change $\Delta P$ at yield change $\Delta y$
by $\Delta P = -P\Delta y D_{\textrm{mod}}$. A zero-coupon bond will have $P = Par/(1+y)^n$ and
$$ D_{\textrm{Mac}} = \frac{1}{P}\frac{nPar}{(1+y)^n} = \frac{1}{P} nP = n, $$
which is most sensitive to $\Delta y$.
Taylor expansion of price function
$$ P(y+\Delta y) = P(y)
				 + \frac{\partial P}{\partial y}\Delta y
				 + \frac{1}{2}\frac{\partial^2 P}{\partial y^2}\Delta y^2
$$
where the second order term accounts for the convexity (as yield increases, $P$
is less sensitive to $\Delta y$). Define dollar convexity
$$ C = \frac{\partial^2 P}{\partial y^2}
     = \sum_{i=1}^n \frac{i(i+1)C}{(1+y)^{i+2}} + \frac{n(n+1)Par}{(1+y)^{n+2}}, $$
then price change is
$$ \Delta P = -P\Delta y D_{\textrm{mod}} + \frac{1}{2}C\Delta y^2 + o(y^3). $$

Numerical approximation of duration and convexity:
\begin{align*}
D_{\textrm{mod}} &= - \frac{P(y+\Delta y) - P(y - \Delta y)}{2P\Delta y} \\
C &= \frac{P(y+\Delta y) - 2P(y) + P(y-\Delta y)}{\Delta y^2}
\end{align*}
Weakness of duration: It only measure interest rate risk, not yield curve risk
(i.e., if yield curve changes shape, not only shifted).

\section{Probability \& Statistics}
Correlation between two RVs $X$ and $Y$:
$$ r_{X,Y} = \frac{\sum_{i=1}^n (X-\bar{X})(Y-\bar{Y})}{(n-1)\sigma_X\sigma_Y} \in [-1, 1]$$
which $\rho_{X,Y}=r_{X,Y}^2\in[0,1]$ is the \emph{coefficient of determination}.

Sum of RVs: Let $Z = w_1X+w_2Y$, and $\Cov(X,Y) = \rho_{X,Y}\sigma_X\sigma_Y$ be the covariance,
\begin{align*}
\E(Z) &= w_1\bar{X} + w_2\bar{Y} \\
\Var(Z) &= w_1^2 \sigma_X^2 + w_2^2 \sigma_Y^2 + 2w_1w_2 \Cov(X,Y)
\end{align*}

Lognormal distribution: Common to model stock price

Wiener process: $W_t+u - W_t \sim N(0,u)$, i.e., standard deviation scales with sq root of time

\subsection*{Regression}

Linear regression: Find line $y=a+bx$ that is best linear unbiased estimator
(BLUE) that relates $x_i$ to $y_i$. Error of regresion: SSR (sum of sq error
due to regression), SSE (sum of sq errors), SST (sum of sq total error, equiv to
sample variance times sample size):
\begin{align*}
SSR &= \sum_i (y(x_i) - \bar{y})^2 \quad \textrm{with }\bar{y} = \frac{1}{n}\sum_i y_i \\
SST &= \sum_i (y_i - \bar{y})^2 \\
SSE &= SST - SSR \\
R^2 &= SSR/SST \in [0,1]
\end{align*}
$R^2$ measures the goodness of fit, i.e., percentage of variation of the data explained by the
regression equation.

Test for \emph{significance} of regression coefficients $a$ and $b$: by
$t$-statistics. Simple linear regression on $n$ samples, the degree of freedom
is $n-2$ for two coefficients are derived from them. With desired level of
significance $\alpha$, a corresponding $t$ value ($t_{\alpha}$) is set. We test
if $|t'| > t_{\alpha}$ to determine the regression coefficient is significant,
i.e., the data is not random. Which (usually $\tilde{a}=0$ as null hypothesis)
\begin{align*}
t' &= \left|\frac{r_{X,Y}\sqrt{n-2}}{\sqrt{1-r_{X,Y}^2}}\right| & & \textrm{for sig of correlation coeff $r_{X,Y}$} \\
t' &= \left|\frac{a-\tilde{a}}{s_a}\right| & & \textrm{for sig of regression coeff $a$ (or similarly $b$)}
\end{align*}

\subsection*{Time series}

Regression on time series: $AR(n)$, auto-regression of lag $n$ periods. For
$AR(1)$, the linear model is $Y_t = a + bY_{t-1}$, i.e. value at $t$ depends
only on value at $t-1$.

Random walk: $AR(1)$ model of $Y_t = a + b Y_{t-1} + \epsilon$ where error term
$epsilon \sim N(0,\sigma^2)$ is independent of any $Y_{\tau}$. Here $b$ is the
\emph{drift} term,
\begin{itemize}
\item if $b\in[0,1)$, the series is \emph{mean-reversion} (pull $Y_t$ back to
the mean). Mean of $Y_t$ is the solution to $\mu = a + b \mu$ or $\mu = a/(1+b)$
\item if $b\in(-1,0]$, the series is oscillating, with the same mean as above
\item if $|b|=1$, the series is a Weiner process
\item if $|b|>1$, the series is explosive
\end{itemize}

\section{Derivatives}

\subsection*{Forward contract}

Risk-free rate $r$, spot price $S_t$, price of forward contract in no-arbitrage
argument:
$$ F_{t,T} = S_t e^{r(T-t)} $$
Forward price should equal to future value of spot price. The price of forward
contract can be extended to include the cost of carry $s$, i.e., all storage,
insurance, etc.: $F_{t,T} = (S_t + s)e^{r(T-t)}$.

Forward contract: No up-front cost, settled at maturity, but no clearing house
to regulate or monitor, with exposure to counterparty credit risk.

\subsection*{Future contract}

Standardized contract, with clearing house. Counterparties deposit an initial
margin, with position marked to market every day and investors shall keep
balance above the maintenance margin.

Spot-future parity (no-arbitrage argument for pricing):
$$ F_{t,T} = (S_t + s) e^{r(T-t)} $$
extension to account for discrete dividents $\delta_i$ paid at times $t_i$:
$$ F_{t,T} = (S_t + s) e^{r(T-t)} - \sum_{i=1}^n \delta_i e^{r(T-t_i)} $$
or with continuous dividend $\delta$:
$$ F_{t,T} = (S_t + s) e^{(r-\delta)(T-t)} $$

Example: interest rate parity. Currency exchange rate $E_t$ at spot and forward
rate $F_{t,T}$ at $T$. Domestic interest rate $r_D$ and foreign interest
rate $r_F$. No arbitrage argument says that if one country's interest rate is
higher than the other, it is balanced by the exchange rate. Then forward rate
is $$F_{t,T} = E_t e^{(r_D - r_F) (T-t)}.$$

Example: implied interest rate. Given future $F_{t,T}$ and spot $S_t$ are known,
we can invert for the interest rate $$r = \frac{1}{T-t} \ln\left(\frac{F_{t,T}}{S_t+s}\right).$$

\subsection*{Options}

Options are traded for hedging, for speculation, and by arbitrageurs.

Put-call parity: Hedge a stock by long a put option and short a call
option with prices $S$, $p$, $c$ respectively, both options with strike price $K$ and
expiry $t$. Stock and put option are financed by borrowed money $Ke^{-rt}$ at
rate $r$ and the sale of call option:
$$ S + p - c = Ke^{-rt} $$
\marginnote{pic at p.100}
Put-call parity can be solved to create synthetic position:
\begin{itemize}
\item put position = borrowing money, long call and short stock\\
    $p=Ke^{-rt} + c - S$
\item short on stock = lending money, long put and short call\\ $-S = p - c - Ke^{-rt}$
\item long on stock = borrow money, short put and long call\\ $S = c - p + Ke^{-rt}$
\end{itemize}
Combining options:
\begin{itemize}
\item Naked call: selling call without owning the asset, payoff to the writer
      is $c-\max(S-K,0)$
\item Naked put: similarly, payoff to the writer is $p-\max(K-S,0)$
\item Straddle: buy both put and a call with same $K$ and expiry, payoff
	  function: $\max(K-S,0) + \max(S-K,0) - c - p$, will lose money (part of
	  premium) if asset price close to $K$ at expiry
\item Naked straddle: sell both put and call with same $K$ and expiry, payoff
      is reverse of straddle
\item Calendar spread: buy call at strike $K$ with expiry $t_1$ and sell call
      at $K$ with expirt $t_2$, use the proceed from sales to finance purchase
\item Bull spread: Long a call with low strike $K_1$ and short a call with
	  higher strike $K_2$ at the same expiry, use the proceed from sale to
	  finance purchase and produce a profit if stock price rises.
\end{itemize}
Nick Leeson \& Barings Bank: Naked straddle on Nikkei 225 at end of 1994.

\subsection*{Option Valuation}
Wiener process for variable $X_t$: $X_t = \mu t + \sigma B_t$ with $\mu$ and
$\sigma^2$ the drift rate and variance rate respectively, $B_t \sim N(0,t)$.

Ito's lemma: for $f(t,X_t)$, which $dX_t = \mu_t dt + \sigma_t dB_t$:
\begin{align*}
df &= \frac{\partial f}{\partial t} dt + \frac{\partial f}{\partial x} dx + \frac{1}{2} \frac{\partial^2 f}{\partial x^2}) dx^2 + O(dt^2) \\
&= \left(\frac{\partial f}{\partial t} + \mu_t \frac{\partial f}{\partial x}+\frac{\sigma_t^2}{2}\frac{\partial^2 f}{\partial x^2}\right) dt + \sigma_t \frac{\partial f}{\partial x}dB_t.
\end{align*}

Asset price model: $dS/S = \mu dt + \sigma dB_t$ with $dB_t^2 \approx dt$,
\begin{align*}
dS &= \mu S dt + \sigma S dB_t \\
dS^2 &= (\mu S dt + \sigma S dB_t)^2 \\
     &= \mu^2 S^2 dt^2 + 2\mu \sigma S^2 dB_t dt + \sigma^2 S^2 dB_t^2 \\
	 &\approx \sigma^2 S^2 dt
\end{align*}
Assume option price $V$ be a function of $S$ and $t$, using Ito's lemma,
\begin{align*}
dV
&= \left(\mu S \frac{\partial V}{\partial S} + \frac{1}{2}\sigma^2 S^2 \frac{\partial^2 V}{\partial S^2} + \frac{\partial V}{\partial t}\right) dt + \sigma S \frac{\partial V}{\partial S} dB_t
\end{align*}
Construct a portfolio $\Pi$ with option $V$ and asset $S$ of quantity $-\Delta$,
\begin{align*}
\Pi &= V - \Delta S \\
d\Pi &= dV - \Delta dS = dV - \Delta(\mu S dt + \sigma S dB_t)\\
&= \left(\mu S \frac{\partial V}{\partial S} + \frac{1}{2}\sigma^2 S^2 \frac{\partial^2 V}{\partial S^2} + \frac{\partial V}{\partial t} - \Delta\mu S\right) dt + \sigma S \left(\frac{\partial V}{\partial S} - \Delta\right) dB_t
\end{align*}
We set $$\Delta = \frac{\partial V}{\partial S}$$ (``delta'', change in value
of option for a change in value of the underlying asset) and $d\Pi = \Pi r dt$
to make the portfolio deterministic, and in a no-arbitrage argument with
risk-free rate $r$, then we have the Black-Scholes model in p.d.e.
\begin{align*}
d\Pi
= \left(\mu S \frac{\partial V}{\partial S} + \frac{1}{2}\sigma^2 S^2 \frac{\partial^2 V}{\partial S^2} + \frac{\partial V}{\partial t} - \Delta\mu S\right) dt
&= \left(V - \frac{\partial V}{\partial S} S\right)rdt \\
\therefore\quad 
\mu S \frac{\partial V}{\partial S} + \frac{1}{2}\sigma^2 S^2 \frac{\partial^2 V}{\partial S^2} + \frac{\partial V}{\partial t} - \frac{\partial V}{\partial S}\mu S
&= Vr - \frac{\partial V}{\partial S} Sr \\
\frac{\partial V}{\partial t} + \frac{1}{2}\sigma^2 S^2 \frac{\partial^2 V}{\partial S^2} + rS \frac{\partial V}{\partial S} - rV &= 0
\end{align*}
The model assumes:
\begin{itemize}
\item European option
\item investor can borrow or lend at rate $r$
\item asset volatility remains constant
\item no dividends paid, no transaction costs, short sales allowed
\end{itemize}
If dividends $r_D$ are paid in the asset, replace $rS$ part with $(r-r_D)S$.

\subsection*{Solving Black-Scholes equation}

European call/put at expiry: $V = \max(S-K,0)$, or value of put $V=\max(K-S, 0)$.
Then the value of a call/put is
\begin{align*}
c(S,t) &= SN(d_1) - Ke^{-rt}N(d_2) \\
p(S,t) &= Ke^{-rt}N(-d_2) - SN(-d_1) \\
\text{where }\qquad
N(d) &= \frac{1}{\sqrt{2\pi}}\int_{-\infty}^d e^{-s^2/2} ds \\
d_1 &= \frac{\log(S/K)+(r+\sigma^2/2)t}{\sigma\sqrt{t}} \\
d_2 &= \frac{\log(S/K)+(r-\sigma^2/2)t}{\sigma\sqrt{t}}
\end{align*}
Solving $\sigma^2$ from option price = Implied volatility.

Proof: With $\frac{\partial d_1}{\partial S}$ and $\frac{\partial d_2}{\partial
S}$, fit $\frac{\partial c}{\partial t}$, $\frac{\partial c}{\partial S}$, and
$\frac{\partial^2 c}{\partial S^2}$ into the Black-Scholes p.d.e. with boundary
condition at $\lim_{S\to 0} c(S,t) = 0$ and
$\lim_{S\to\infty} c(S,t) = S - Ke^{-rt}$. Then derive $p(S,t)$ from $c(S,t)$
from put-call parity. $\square$

Delta of call and put are respectively, $\frac{\partial c}{\partial S} = N(d_1)$
and $\frac{\partial p}{\partial S} = N(d_1) - 1$

Black-Scholes model is good at pricing options at the money but less so for out
of the money and deep in the money options.

Risk-neutrality approach (?): Assume asset price modeled with drift at interest rate $r$,
$dS_t = rS_t dt + \sigma S_t dB_t$. Then the density function of future values of $S_t$
$$ \Pr[S_t = s] = \frac{1}{\sigma s \sqrt{2\pi t}} \exp\left(-\frac{\left(\log(\frac{s}{S_0}) - (r-\frac{1}{2}\sigma^2)t\right)^2}{2\sigma^2t}\right)$$
With payoff function $V(S_T,T)$ at time $T$, the expected payoff is
$$ E[V(S_T,T)] = \frac{1}{\sigma \sqrt{2\pi T}} \int_0^{\infty} \frac{V(s)}{s} \exp\left(-\frac{\left(\log(\frac{s}{S_0}) - (r-\frac{1}{2}\sigma^2)T\right)^2}{2\sigma^2T}\right) ds $$
discounting to present value:
$$ E[V(S_0,0)] = \frac{e^{-rT}}{\sigma \sqrt{2\pi T}} \int_0^{\infty} \frac{V(s)}{s} \exp\left(-\frac{\left(\log(\frac{s}{S_0}) - (r-\frac{1}{2}\sigma^2)T\right)^2}{2\sigma^2T}\right) ds $$
Which satisfies the p.d.e.

\subsection*{Binomial tree method}

For handling path-dependent options and situations that Black-Scholes equation
cannot handle, e.g., interest rate options. Consider asset price $S$ at discrete
time period $\Delta t$, the probability of upward ($S\to Su$) and downward
($S\to Sd$) move are $\pi_u$ and $\pi_d$ respectively, $\pi_u+\pi_d = 1$.  To
make the model fit $dS=\mu Sdt + \sigma S dB_t$,
\begin{align*}
    u &= e^{\sigma\sqrt{\Delta t}} &
    d &= \frac{1}{u} &
\pi_u &= \frac{e^{r\Delta t} - d}{u-d} &
\pi_d &= 1=\pi_u = \frac{u - e^{r\Delta t}}{u-d}
\end{align*}
Algorithm:
\begin{enumerate}
\item set up a tree with $S_0$ at root, each time step branch into upward and downward price until expiry
\item at leaf nodes, derive option value based on the price
\item trace backward in time the expected option value, using the upward and downward probabilities
\item expected option value at root = option price
\end{enumerate}

Trinomial method is similar, with possibility of sideway move at probability $\pi_m$:
\begin{align*}
  \E[dS/S] &= \pi_u u + \pi_m (0) + \pi_d d = \mu \Delta t \\
\Var[dS/S] &= \pi_u u^2 + \pi_m (0) + \pi_d d^2 = \mu \Delta t^2 + \sigma^2 \Delta z^2 \approx \sigma^2 \Delta t \\
        1 &= \pi_u + \pi_m + \pi_d
\end{align*}
which gives the following:
\begin{align*}
    u &= e^{\sigma\sqrt{2 \Delta t}} &
    d &= \frac{1}{u} = e^{-\sigma\sqrt{2\Delta t}} \\
\pi_u &= \left(\frac{e^{r\Delta t/2} - e^{-\sigma\sqrt{\Delta t/2}}}
                    {e^{\sigma\sqrt{\Delta t/2}} - e^{-\sigma\sqrt{\Delta t/2}}}\right)^2 &
\pi_d &= \left(\frac{e^{\sigma\sqrt{\Delta t/2}} - e^{r\Delta t/2}}
                    {e^{\sigma\sqrt{\Delta t/2}} - e^{-\sigma\sqrt{\Delta t/2}}}\right)^2 \\
\pi_m &= 1-\pi_u-\pi_d
\end{align*}

\subsection*{Finite difference method}

Solve Black-Scholes p.d.e. by discretization: option value $V(S,t)$, solved on
grid points of $(S,t)$ for $S\in[S_{\min},S_{\max}]$ and $t\in[0,T]$ with
increments $\Delta S$ and $\Delta t$ respectively. We substitute with finite
difference form of partial derivatives:
\begin{align*}
\frac{\partial V}{\partial t} & \approx \frac{V(S,t+\Delta t)-V(S,t)}{\Delta t} \\
\frac{\partial V}{\partial S} & \approx \frac{V(S+\Delta S,t)-V(S-\Delta S,t)}{2\Delta S} \\
\frac{\partial^2 V}{\partial S^2} & \approx \frac{V(S+\Delta S,t) - 2V(S,t) + V(S-\Delta S,t)}{\Delta S^2}
\end{align*}
into the p.d.e. and get
\begin{align*}
\frac{\partial V}{\partial t} +
\frac{1}{2}\sigma^2S^2\frac{\partial^2V}{\partial S^2}+
rS\frac{\partial V}{\partial S}-
rV &= 0 \\
\frac{V(S,t+\Delta t)-V(S,t)}{\Delta t} + 
\frac{1}{2}\sigma^2S^2\frac{V(S+\Delta S,t)-2V(S,t)+V(S-\Delta S,t)}{\Delta S^2} & \\ + 
rS\frac{V(S+\Delta S,t)-V(S-\Delta S,t)}{2\Delta S}-rV(S,t) &= 0 \\
\end{align*}
Solve for $V(S,t+\Delta t)$ gives a system of linear equations of $V(\cdot,t)$:
\begin{align*}
V(S,t+\Delta t) &= a V(S-\Delta S,t) + b V(S,t) + c V(S+\Delta S,t) \\
\textrm{where}\qquad
a &= -\frac{1}{2}\sigma^2\Delta t\frac{S^2}{\Delta S^2} + \frac{r\Delta t S}{2\Delta S} \\
b &= 1+\frac{\sigma^2\Delta t S^2}{\Delta S^2} + r\Delta t \\
c &= -\frac{1}{2}\sigma^2\Delta t\frac{S^2}{\Delta S^2} - \frac{r\Delta t S}{2\Delta S}
\end{align*}
Boundary conditions:
\begin{itemize}
\item for a call option, $S << K$, $V(S,t)=0$ and $V(S,t)=S$ as $S\to\infty$
\item for a put option, $S >> K$, $V(S,t)=0$ and $V(S,t)=K$ as $S\to 0$
\item at expiry, $V(S,t)=\max(S-K,0)$ for a call and $V(S,t)=\max(K-S,0)$ for a put
\end{itemize}

Solution by matrix: Compute $V(S,t)$ on a grid such that $S=S_{\min} + i\Delta S$
and $t=t_0 + j\Delta t$ with $i=0,\cdots,M$ and $j=0,\cdots,N$. Then the above
equation will become
$$ a_iV_{i-1,j-1} + b_iV_{i,j-1} + cV_{i+1,j-1} = V_{i,j} $$
And $V_{\cdot,j-1}$ is related to $V_{\cdot,j}$ by a tridiagonal matrix (which
can be LU-factorized). We can derive $V(S,t_0)$ backward from $V(S,T)$ one step
at a time by matrix multiplication.

Solution by explicit derivation: The p.d.e. can be alternatively written as
\begin{align*}
\frac{V_{i,j+1}-V_{i,j}}{\Delta t} + 
    \frac{1}{2}\sigma^2S^2\frac{V_{i+1,j+1}-2V_{i,j+1}+V_{i-1,j+1}}{\Delta S^2} & \\ + 
    rS\frac{V_{i+1,j+1}-V_{i-1,j+1}}{2\Delta S}-rV_{i,j} &= 0
\end{align*}
by evaluating $\frac{\partial V}{\partial S}$ and $\frac{\partial^2V}{\partial S^2}$
at $t = t_0+(j+1)\Delta t$. Solving for $V_{i,j}$ term
\begin{align*}
V_{i,j} &= d_i V_{i-1,j+1} + e_i V_{i,j+1} + f_i V_{i-1,j+1} \\
\textrm{where}\qquad
d_i &= \frac{1}{1+r\Delta t}\left(
    -\frac{1}{2} r\Delta t\frac{S_{\min} + i\Delta S}{\Delta S}
    +\frac{1}{2} \Delta t \sigma^2 \left(\frac{S_{\min} + i\Delta S}{\Delta S} \right)^2
\right) \\
e_i &= \frac{1}{1+r\Delta t}\left(
    1-\Delta t\sigma^2 \left(\frac{S_{\min} + i\Delta S}{\Delta S} \right)^2
\right) \\
f_i &= \frac{1}{1+r\Delta t}\left(
    \frac{1}{2} r\Delta t\frac{S_{\min} + i\Delta S}{\Delta S}
    +\frac{1}{2} \Delta t \sigma^2 \left(\frac{S_{\min} + i\Delta S}{\Delta S} \right)^2
\right)
\end{align*}

\subsection*{Monte Carlo analysis}
Algorithm:
\begin{enumerate}
\item Generate random walks (usually lognormal) from $t$ to expiry with increment $\Delta t$
\item At the end of each path, evaluate the payoff at expiry
\item Compuate average of all such payoff, and discount to PV by $e^{-rt}$
\end{enumerate}

Model stock price with lognormal distribution:
\begin{align*}
    df &= \sigma S \frac{\partial f}{\partial S} dB_t + \left(\mu S
    \frac{\partial f}{\partial S} + \frac{1}{2}\sigma^2S^2\frac{\partial^2
f}{\partial S^2}\right)dt \\
\intertext{with $f=\log S$:}
d(\log S) &= \sigma S \frac{1}{S} dB_t + \left(\mu S
\frac{1}{S} + \frac{1}{2}\sigma^2S^2(-\frac{1}{S^2})
\right)dt \\
&= \sigma dB_t + \left(\mu - \frac{1}{2}\sigma^2 \right)dt \\
\log S_T - \log S_0 &= \int_{t=0}^{t=T} \sigma dB_t + \int_0^T (\mu - \frac{1}{2}\sigma^2)dt \\
S_T &= S_0 \exp\left((\mu-\frac{1}{2}\sigma^2)T + \int_{t=0}^{t=T}\sigma dB_t
\right) \\
\implies\qquad \Delta S &= \exp\left((\mu-\frac{1}{2}\sigma^2)\Delta t + \sigma\sqrt{\Delta t} z \right)
\end{align*}
with $z\sim N(0,1)$. Random walk is generated from steps $Delta S$ of random size.

\subsection*{Option greeks}
\begin{align*}
\Delta &= \frac{\partial V}{\partial S} &
\Gamma &= \frac{\partial^2 V}{\partial S^2} &
\theta &= \frac{\partial V}{\partial t} &
v &= \frac{\partial V}{\partial\sigma} &
\rho &= \frac{\partial V}{\partial r}
\end{align*}
Delta $\Delta$ always have positive slope, with call $\Delta\ge0$ and put
$\Delta\le0$. Call's delta is 0 on extremely low stock price and 1 on extremely
high price, for option is worthless if deep out of money and
$\displaystyle\lim_{S\to\infty} V=S$ if deep into the money.

Gamma $\Gamma$ is the rate of change of delta, always positive. Gamma for call
and put are the same.

Theta $\theta$ is most sensitive when the option is close to the money, with
sensitivity decreasing as the option approaches expiry.

Vega $v$ is the option sensitivity to change in volatility. Vega for call and
put are the same.

Rho $\rho$ is the sensitivity of option price to interest rate.

\section{Fixed income}
Bond equivalent yield is computed on the basis of a 365-day year.

Probability of default estimation: A corporate bond with yield $y$ has
probability of default $p$, which assumed default will lost all value, if a
treasury bond has yield $y'$ and $(1-p)(1+y) = (1+y')$

Par yield: Coupon rate, the yield if priced at par.

Forward yield: Yield of a loan expected for future time.

Spot yield: Yield of the market today. Always between par and forward.

For an upward sloping yield curve (i.e., yield goes up for par yield and spot
yield for longer time or forward yield further in the future), forward $>$ par.

For a downward sloping yield curve, forward $<$ par.

PCA finds that 90\% of price change of bonds was due to changes in interest
levels; 8.5\% due to changing yield curve slope; 1.5\% due to change of
curvature of the yield curve.

Binomial tree method for valuation: Black-Derman-Toy model, Ho-Lee model, Heath-Jarrow-Morton model

Morgage-backed securities: Common to assume an annualized constant prepayment
rate (CPR) and find the equivalent single month mortality rate (SMM):\[
SMM = 1 - (1-CPR)^{1/12}
\]

\section{Equity markets}
Systematic risks: All stocks are subject to

Unsystematic risk: Company-specific portion of the risk

Price of stock calculated using perpetuity model: $P = \frac{DIV}{r_{CE}}$
where $DIV$ is the dividend, expected to remain stable over time, $r_{CE}$ is the
(unobservable) cost of equity capital (e.g. hurdle rate for new projects). If
dividend is expected to grow at constant rate $g$, we have the Gordon growth
model\[
    P = \frac{DIV_0(1+g)}{r_{CE}-g}
\]

Common multiples for analysis:
\begin{itemize}
\item price/earnig ratio
\item dividend/price (aka dividend yield): high dividend yield if price is
    undervalued
\item price/sales
\item price/book value
\item price/cashflow
\item return on asset
\end{itemize}

Capital asset pricing model (CAPM): Compare return of stock $R_s$ to return of
market $R_m$ with assumed risk-free return $R_f$ \[
    R_s - R_f = \beta (R_m - R_f)
\]
where $\beta$ is the measure of correlation between the security and the market.
The return above $R_f$ is the excess return.

Dupont analysis: Calculate the return on equity as
\begin{align*}
ROE &= \frac{\textrm{sales}}{\textrm{assets}}\frac{\textrm{net
    income}}{\textrm{sales}}\frac{\textrm{assets}}{\textrm{equity}} \\
&= (\textrm{asset turnover})(\textrm{profit margin})(\textrm{leverage ratio})
\end{align*}

Markowitz efficient frontier: Return vs Risk cluster plot will show a boundary
of max return on risk. The boundary is concave, monotonically increasing on
increasing risk.

\subsection*{Valuting convertible bonds}
Convertible bond has a term sheet to specify a conversion price and conversion
ratio. Most convertible bonds are callable, allowing the issuer to force
conversion if so desired.

\begin{enumerate}
\item Determine the min value of the convertible bond (comparing the bond as a
    straight bond without conversion feature)
\item Calculate the investment premium
\item Calculate the premium payback period
\item Calulate value of the embedded convertible bond option
\end{enumerate}
Black-Scholes equation can be used to value option on the underlying stock but
not be used to value the embedded call. Bonds do not have constant volatilities
as BS assumed, but descreasing volatilities as bonds approach maturity, for
holder receives a known par at maturity. Also bonds would never be worth more
than its par value plus any accrued coupons, thus bonds does not fit a lognormal
distribution which price of zero to infinity are possible.

Example:
\begin{enumerate}
\item 8-yr 7\% coupon convertible bond, compare to 8\% yield in market, its
par should be $\sum_{k=1}^{16} \frac{7}{2}(1.04)^{-k} + 100(1.04)^{-16} = 94.17$.
The \$100 bond are convertible to 5.263 shares of stock when it hit \$19, which
the conversion ratio is $k=100/19=5.263$. If the current stock price is \$15.5,
the conversion value of the bond is $15.5k=81.576$. Taking the max of the two,
the straight bond should worth \$94.17.
\item If the market price for this convertible bond is \$120, the premium is
$\frac{120}{94.17}-1 = 27\%$ over the market price of the bond. Alternatively,
the expected price for stock conversion is $\frac{120}{5.263}=\$22.8$ per
share, which is $\frac{22.8-15.5}{15.5}=47\%$ premium over the market price of
the shares.
\item A fair value is which the coupon/dividend interest
should offset the premium we pay to hold the bond. Here the shares premium over
the coupon is $\frac{120-5.263(15.5)}{7} = 5.48$ years. Normally we expect this
to be 3--5 years.
\item Value of the embedded call: price of convertible bond = price of straight
bond + call option. Say, $T=8$, $S=15.5$, $K=19$, $r_f=5\%$, $\sigma=35\%$,
using Black-Scholes, the call option per share worth $6.835$. Thus the price of
the convertible bond should be $94.17 + 6.835(5.263) = 130.1$. If the issuer
can also call the bond at any time after 3 years, we should subtract the value
of issuer's call option.
\end{enumerate}

\section{Risk management}
Value at Risk (VaR): the economic loss that can be expected at a given
confidence level under adverse market condtions, calculated using simulation or
distribution

To measure VaR, first evaluate all possible risk exposures and choose a
confidence level (e.g. 95\% or 99\%), and a time horizon. Then the "worst case"
value is\[
VaR = V_0 \alpha \sigma \sqrt{t}
\]
where $\alpha$ is the inv of std normal distribution at the selected confidence
interval, $\sigma$ is the asset's std dev, $t$ is the time scale factor between
$\sigma$ and VaR (equity takes $t=250$ to convert daily volitility to annual)
and $V_0$ to the spot price.

\appendix

\section{Formulae}
Taylor series
$$ f(x) = \sum_{n=0}^\infty \frac{f^{(n)}(k)}{n!} (x-k)^n = f(k) + f'(k)(x-k) + \frac{f''(k)}{2!}(x-k)^2 + \cdots $$
Multivariate Taylor series:
\begin{align*}
df(x,y) &= f(x+dx,y+dy) - f(x,y) \\
        &=	\frac{\partial f}{\partial x}dx +
		 	\frac{\partial f}{\partial y}dy +
			\frac{1}{2!} \left(
				\frac{\partial^2 f}{\partial x^2}dx^2 +
				2\frac{\partial^2 f}{\partial x\partial y}dxdy +
				\frac{\partial^2 f}{\partial y^2}dy^2
			\right) + \cdots
\end{align*}
Variance scaling: $Var(\alpha X) = \alpha^2 Var(X)$

CLT: $\displaystyle Z=\frac{\bar{X}-\mu_X}{\sigma_X/\sqrt{n}}$ is normally distributed regardless the distro of $X$

Lognormal PDF: $\displaystyle f(x) = \frac{1}{\sqrt{2\pi\sigma^2x^2}}\exp\left(-\frac{(\ln x-\mu)^2}{2\sigma^2}\right)$
\end{document}

Notes
-----
<http://banach.millersville.edu/~bob/book/BlackScholes/main.pdf>
<http://www.math.cuhk.edu.hk/~rchan/teaching/math4210/chap08.pdf>
<http://www.math.uaic.ro/~annalsmath/pdf-uri%20anale/F1(2010)/Mosneagu.pdf>
<http://www.iam.fmph.uniba.sk/institute/stehlikova/fd14en/lectures/05_black_scholes_1.pdf>
https://www.math.upenn.edu/~deturck/m425/m425-blackscholessol-16a.pdf
http://www.math.unl.edu/~sdunbar1/MathematicalFinance/Lessons/BlackScholes/Solution/solution.pdf
http://www.ntu.edu.sg/home/nprivault/MA5182/black-scholes-pde.pdf

